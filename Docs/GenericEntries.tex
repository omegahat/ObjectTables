\documentclass{article}
\usepackage{fullpage}
\usepackage{times}

\input{WebMacros}
\input{pstricks}
\input{SMacros}
\input{CMacros}

\title{External Values as R Variables: Attaching arbitrary data to R's
  search path}
\author{Duncan Temple Lang}
\date{}

\begin{document}
\maketitle
\begin{abstract}
  
This document outlines changes to the internal code for elements of
the search path and generally databases in the S4 sense so that users
can define their own database types.  The model follows the S4
approach in which each database ``class'' or type must support five
(5) methods: exists, get, assign, remove and objects.  The motivation
is to allow databases providing name-value bindings for values that
are external to R.  These include relational databases and tables;
tables from other languages and systems such as Omegahat, Java, Perl,
Python, etc.; naming services such as CORBA; archives such as zip and
tar files; and so on.
The basic model mimics that defined in S4 \cite{S4:databases},
and we add a dynamic and lazy  lookup mechanism to aid
with external synchronization.

\end{abstract}

\section{Motivation}

External data sources.
Provide name-value bindings.
The values are external objects that need not be brought
into R until the last minute.

Database tables:
\begin{verbatim}
  attach(dbTable())
  col1
      gets reference to col1
  mean(col1) 
      select mean(col1) from table;
  length(col1)
      select count(col1) from table;
\end{verbatim}


Live-variables from foreign embedded systems: Perl, Python, Java, JavaScript.
   (Matlab/Octave)

Static/Serialized objects from foreign systems:

CORBA

Archive files: Tar, zip

URIs: FTP \& HTTP directories

Methods for dispatching on databases.

\section{Dynamic versus Static Object Lists}

Lazy versus eager object listing. \\
Infinite recursion.

Listeners: polling versus event notification.
 CORBA events, GTK, directory stat, RDBMS triggers,
\\
 threads, file descriptors, $\ldots$

Static can be implemented using onAttach

\section{Database Interface}

An object that is to be attached must implement the interface
``Database''. This means that it must support methods for
\SFunction{exists}, \SFunction{get}, \SFunction{remove},
\SFunction{assign} and \SFunction{objects}.  (The \SFunction{exists}
function can be inherited from the default method which performs a
linear search through the vector returned from a call to
\SFunction{objects}.)

Each database might have a C representation of the following
form
\begin{verbatim}

typedef Rboolean (*Rdb_exists)(const char *name, R_SearchPathElement *);
typedef SEXP     (*Rdb_get)(const char *name, R_SearchPathElement *);
typedef SEXP     (*Rdb_remove)(const char *name, R_SearchPathElement *);
typedef SEXP     (*Rdb_assign)(const char *name, SEXP value, R_SearchPathElement *);
typedef SEXP     (*Rdb_objects)(R_SearchPathElement *);

typedef struct  _R_SearchPathElement{

  int       type;
  char    **cachedNames;

  Rdb_exists  exists;
  Rdb_get     get;
  Rdb_remove  remove;
  Rdb_assign  assign;
  Rdb_objects objects;
  
  void     *privateData;

} R_SearchPathElement;
\end{verbatim}

We don't need the exists.  We do want \Croutine{attach} and
\Croutine{detach} routines.

The \Croutine{onAttach} and \Croutine{onDetach} routines allow one to
perform constructor and destructor actions such as creating a
temporary working directory, caching the names of the objects,
throwing an error if a connection to the remote system is not
available, $\ldots$.



Rather than introduce S-language level methods at this point, we
require each instance of the SearchPathElement structure to provide
routines that perform the dispatching.  We provide generic routines
that will do the dispatching at the S-level. In this way, we have a
class of SearchPathElement entries which are controlled at the user
level. This will be relatively slow, so people will want to specialize
these types of databases where and when it is possible.

\begin{verbatim}
SEXP createUserLevelDatabase(...) {

 R_SearchPathElement *el = (R_SearchPathElement *) malloc(sizeof(R_SearchPathElement));
 el->exists = Rdb_MethodExists;
 el->get = Rdb_MethodGet;
 el->remove = Rdb_MethodRemove;
 el->assign = Rdb_MethodAssign;
 el->objects = Rdb_MethodObjects;

 ...

 return(el);
}
\end{verbatim}


Each database/search path element can cache the names of the variables
it contains using the \Cfield{cachedNames} field.  If the database
sets this field, the R engine will assume the cache is up-to-date and
use its own cache. When the cache is known to be out of date, the
database should unset this value.


\section{Implementation}
In order to be able to use these attached objects in R,
we need to be able to integrate them with the current code
and have the R interpreter treat them as part of the
search path but interact with them differently.

We can Use Luke's external pointer mechanism to handle user-defined
search path elements in a simple way.  The developer of a database
type, say DB, provides a C routine that allocates the necessary C
level structures and initializes them and stuffs them into a C object
of type \CStruct{RObjectTable}. Along with this database specific
class and instance data, they specify the C routines to be used for
the five ($5$) methods defined by the R-Table interface
(\Croutine{get}, \Croutine{exists}, \Croutine{objects},
\Croutine{assign} and \Croutine{remove}). Again, these are fields in
the \CStruct{RObjectTable}.)  We wrap the \CStruct{RObjectTable}
instance into an R object. This is done using the
\Croutine{R_MakeExternalPtr} routine and passed back to the R user as
a ``regular'' R object.

In principle, we could attach this R object directly.  Then in the C
code that manipulates such objects (e.g. \Croutine{do_ls} associated
with \SFunction{objects}, \Croutine{findVarInFrame} associated with
\SFunction{get}, etc.), we would recognize the object as being of a
particular and special type (e.g. having a class \SClass{Database})
and handle it by calling its internal method routine.

The only difficulty with this approach is that we also have to
consider how the database co-exists with the other search path
elements and how the R engine connects these.  Specifically, we have
to deal with the enclosing environment of our user-defined table. R
wants to check in one table and then in its parent table and so on
when looking for variables.  We need to play consistently with this
mechanism.


What we propose is that we use the 

\end{document}

